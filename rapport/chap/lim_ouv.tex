Vous me permettrez dans cette dernière partie d'être un peu plus personnel. 

 \section{Limites}

Si ce fut très instructif, nous sommes aussi conscient des limites de ce programme. \\

Parmi elles : 
\begin{itemize}
\item Nous aurions aimé le tester sur tous les fichiers de nos deux corpus. Mais comme dit dans l'interface, cela prendrait au moins 24 à 48 heures. Au moment où le programme fut 'terminé', il ne nous restait pas ce temps-là. \\

\item Nous avons du pour enlever manuellement toutes les informations des fichiers de \emph{gutemberg.org}. Nous sommes conscient que cela ne devrait pas être le cas, d'où le fait de le mentionner explicitement, par respect pour leur travail. \\

\item Nous n'avons toujours pas régler le problème, et de la ponctuation, et de la définition de mots, phrases. Nous avons fait un choix pour pouvoir avancer, mais la question mérite réflexion. \\

\item Enfin, nous aurions aimé affiner le programme, le rendre plus interactif, proposer plus d'options... y travailler encore plus.
\end{itemize}

\section{Ouverture et réflexions diverses}

Malgré ce que nous avons dit plus haut, ce fut une expérience très enrichissante, et pas juste au niveau académique. \\

Si j'ai souvent programmé pour le plaisir, pour la première fois, j'avais un délai, un rapport à rendre et donc un peu plus de 'bonne' pression. \\

Parmi les points les plus marquants :
\begin{itemize}
\item Nous avons consacré beaucoup d'heures (plusieurs dizaines voire une centaine) à réfléchir, essayer, coder, mais chaque petit pas fut une grande récompense.

\item Ce fut un exercice d'humilité et de lucidité.

D'humilité, car par définition, à chaque lancement, on s'attend à voir le programme ne pas marcher. Au départ, on s'agace car l'égo en prend un coup, puis on trouve cela normal, et on finit même par bien le prendre, et en faire un défi de plus à relever\footnote{sauf si c'est le jour de la présentation. \Smiley}.

De lucidité, car on est obligé de bien s'exprimer. Le programme ne fait pas ce que l'\emph{on veut qu'il fasse}, mais plutôt ce qu'\emph{on lui dit de faire}. Il ne sert donc à rien d'invoquer le ciel\footnote{sauf pour avoir l'inspiration.} ou de s'énerver, mais plutôt il nous faut réfléchir à ce que l'on a donné comme instruction, et non ce que l'on voulait dire. \\

\item Enfin, et nous terminerons par ce point, le meilleur conseil que nous ayons reçu depuis fort longtemps nous a été donné par Mr Paroubek \textbf{«Testez votre programme sur un corpus.»} Nous ne sommes pas prêts de l'oublier. \\

Nous étions à mi-chemin lorsque ce conseil nous a été donné. Si dans la théorie, toutes nos fonctions faisaient leur travail, et s'appliquaient bien aux phrases-test que nous avions créées, il n'y en a pas une qui ait résisté au test sur corpus. \\

Pendant deux jours, nous avons retouché, voire recodé toutes les fonctions, mais mieux vaut cela que de se rendre au dernier moment que rien ne marche, après y avoir consacré beaucoup de temps et d'énergie. \\

Ainsi, dans l'informatique comme dans la vie, si la théorie a sa place, rien ne vaut l'expérience pratique. \\
\end{itemize}











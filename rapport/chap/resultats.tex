\section{Fichier xml final}

Le résultat du programme est inscrit dans le fichier xml qui se présente ainsi : \\ 

\fbox{
\begin{minipage}{.85\textwidth}
\footnotesize{

\texttt{<?xml version="1.0" encoding="UTF-8" standalone="no" ?>}

\texttt{<!DOCTYPE banque\_donnees SYSTEM "synthese\_xml\_total.dtd">} \\

\texttt{<banque\_donnees> } 

\texttt{\hspace{.5cm}<corpus type\_corpus='corpus\_*'> } 

\texttt{\hspace{1cm}<text text\_id='c.t.'> } 

\texttt{\hspace{1.5cm}<sentence sentence\_id='c.t.s.'> } 

\texttt{\hspace{2cm}<word word\_id='c.t.s.w.' word\_form='xxx'> } 

%----------------------------------------------------------------------------
\texttt{\hspace{2.5cm}<morphology> } 

\texttt{\hspace{2.8cm}<lemme> } 
\texttt{\hspace{1cm}<![CDATA[ x ]]>}
\texttt{\hspace{1cm}</lemme> } 

\texttt{\hspace{2.8cm}<POS\_treetagger> } 
\texttt{\hspace{0cm}<![CDATA[ x ]]>}
\texttt{\hspace{0cm}</POS\_treetagger> } 

\texttt{\hspace{2.8cm}<POS\_morphalou> } 
\texttt{\hspace{0cm}<![CDATA[ x ]]>}
\texttt{\hspace{0cm}</POS\_morphalou> } 

\texttt{\hspace{2.8cm}<genre> } 
\texttt{\hspace{0cm}<![CDATA[ x ]]>}
\texttt{\hspace{0cm}</genre> } 

\texttt{\hspace{2.8cm}<nombre> } 
\texttt{\hspace{0cm}<![CDATA[ x ]]>}
\texttt{\hspace{0cm}</nombre> } 

\texttt{\hspace{2.5cm}</morphology> } 

%----------------------------------------------------------------------------
\texttt{\hspace{2.5cm}<statistics> } 

\texttt{\hspace{2.8cm}<nb\_apparition\_text> } 
\texttt{\hspace{0cm}<![CDATA[ x ]]>}
\texttt{\hspace{0cm}</nb\_apparition\_text> } 

\texttt{\hspace{2.8cm}<nb\_apparition\_corpus> } 
\texttt{\hspace{0cm}<![CDATA[ x ]]>}
\texttt{\hspace{0cm}</nb\_apparition\_corpus> } 

\texttt{\hspace{2.8cm}<frequence\_in\_text> } 
\texttt{\hspace{0cm}<![CDATA[ x ]]>}
\texttt{\hspace{0cm}</frequence\_in\_text> } 

\texttt{\hspace{2.8cm}<frequence\_in\_corpus> } 
\texttt{\hspace{0cm}<![CDATA[ x ]]>}
\texttt{\hspace{0cm}</frequence\_in\_corpus> } 

\texttt{\hspace{2.5cm}</statistics> } 
%----------------------------------------------------------------------------

\texttt{\hspace{2cm}</word> } 

\texttt{\hspace{1.5cm}</sentence> } 

\texttt{\hspace{1cm}</text> } 

\texttt{\hspace{.5cm}</corpus> } 

\texttt{</banque\_donnees> } 

}
\end{minipage}
}

~~\\

Chaque élément xml (du corpus au mot) peut être multiplié autant de fois que nécessaire. \\

\section{Temps d'exécution}

Nous avons testé le programme avec un nombre de phrases à traiter différent\footnote{Si le nombre de phrases à traiter excède le nombre de phrases contenues dans le fichier, la variable s'ajuste automatiquement au nombre de phrases du fichier}. \\

\begin{tabular}{|c|c|}
\hline
\textbf{Nombre de phrases par fichier} & \textbf{Temps d'exécution approximatif} \\ 
\hline \hline
1 phrase & 3 minutes \\
\hline
10 phrases & 20 minutes \\
\hline
15 phrases & 30 minutes \\
\hline
50 phrases & 1 heure 30 minutes \\
\hline
\end{tabular} 
~\\

%----------------------------------------------------------------------------
\section{Réponse à notre hypothèse initiale}

Comme dit au point 1.3, il nous fallait voir si le corpus de Mr Paroubek passait notre premier texte. \\

Contrairement à ce que l'on pensait, au niveau du nombre de mots par phrase\footnote{qui était notre principal outil de séparation}, la différence n'est pas significative : \\

\begin{tabular}{|c|P{3cm}|P{4cm}|P{3cm}|}
\hline
\textbf{Corpus} & \textbf{Nombre de mots total} & \textbf{Nombre de mots par phrase} & \textbf{Différence} \\ 
\hline \hline
Corpus littérature & 2.570.695 & 22.80 & + 8 \% \\
\hline
Corpus Mr Paroubek & 23.163 & 21.07 & - 8 \% \\
\hline
\end{tabular} 
~\\

Ainsi, basé sur ce critère, le corpus de Mr Paroubek passe le premier test. \\

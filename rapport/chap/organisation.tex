\section{Approche}

%-------------------------------------------------------------
\newpage
\section{Organisation des dossiers}

\fbox{
\begin{minipage}[t]{.4\textwidth}
\footnotesize{ 
\begin{center}
AVANT EXÉCUTION DU PROGRAMME
\end{center}

\begin{itemize}
\item \textbf{PROJECT\_ROOT/}
	\begin{itemize}
	\item {code\_source/}
		\begin{itemize}
		\item {modules/}
		\end{itemize}
	\item {corpus\_litterature/}
	\item {corpus\_professeur/}
	\item {from\_outside\_treetagger/}
	\item {morphalo/}
	\item {rapport/}
	\end{itemize}
\end{itemize}

}
\end{minipage} 
} \fbox{
\begin{minipage}[t]{.4\textwidth}
\footnotesize{ 
\begin{center}
APRÈS EXÉCUTION DU PROGRAMME
\end{center}

\begin{itemize}
\item \textbf{PROJECT\_ROOT/}
	\begin{itemize}
	\item {code\_source/}
		\begin{itemize}
			\item {modules/}
		\end{itemize}
	\item {corpus\_litterature/}
		\begin{itemize}
		\item \textbf{statistiques/}
		\item \textbf{xml/}
		\item \textbf{pdf/}
		\end{itemize}
	\item {corpus\_professeur/}
		\begin{itemize}
		\item \textbf{statistiques/}
		\item \textbf{xml/}
		\end{itemize}
	\item {from\_outside\_treetagger/}
	\item {morphalo/}
	\item {rapport/}
	\item \textbf{resultats\_xml/}
	\end{itemize}
\end{itemize}
}
\end{minipage}
} 

~\\

Les noms nous paraissent assez explicites pour ne pas être tous détaillés. Néanmoins, certains dossiers nécessitent  une petite explication.

	\begin{itemize}
	\item \textbf{from\_outside\_treetagger/} : au début de la programmation, l'accès aux éléments de morphalou était difficile. Pour ne pas perdre de temps, nous avions décidé d'utiliser aussi tree\_tagger.
	
	Celui-ci ayant besoin de fichiers particuliers pour fonctionner, nous les avons regroupés sous ce dossier.
	
	\item \textbf{corpus$_x$/statistiques/} : Contient les statistiques de base, ainsi que les distributions de mots de chaque fichier pris séparément, puis du corpus en entier.
	
	\item \textbf{corpus$_x$/xml/} : Contient l'arbre xml concernant le corpus.
	
	\item \textbf{resultats\_xml} : S'appuyant sur les deux dossiers précédemment cités, ce dossier contient toutes les données que nous avons pu recueillir.
	\end{itemize}

%------------------------------------------------------------
\newpage
\section{Présentation des modules}

La partie programmée est regroupée dans le dossier \emph{code\_source}. \\

 \begin{wrapfigure}{l}{.45\textwidth} % { r, l, i } pour la position,
\fbox{
\begin{minipage}{.40\textwidth}
\footnotesize{
\textbf{code\_source/}
	\begin{itemize}
	\item {main.py}
	\item {settings.py}
	\item {modules/}
		\begin{itemize}
		\item {big\_process.py} \\
		\item {ponctuation\_texte.py}
		\item {tagging.py}
		\item {stats\_0\_distributions.py}
		\item {stats\_1\_base.py}
		
		\item {writing\_in\_files.py}
		\item {others.py}
		\end{itemize}
	\end{itemize}
}
\end{minipage}
} 
\end{wrapfigure}

Cette fois-ci, chaque fichier mérite que l'on vous explicite ce qu'il contient. \\

De même, nous aimerions préciser la logique de notre mouvement (flow). C'est en la suivant que j'ai pu, de manière systématique naviguer à travers le code\footnote{qui devenait de plus en plus gros.}. Je pars toujours du niveau 0 vers le niveau 2. \\  

\underline{Niveau 0}
	\begin{itemize}
	\item \textbf{main.py} : Contient l'interface. C'est le fichier qui englobe le tout.
	\item \textbf{settings.py} : Pour faciliter la gestion des liens, nous avons regroupé les liens vers nos dossiers dans ce fichier. Il ne contient aucune fonction.
	\end{itemize}

\underline{Niveau 1}
	\begin{itemize}
	\item \textbf{modules/big\_process.py} : Ne contient que des fonctions composées d'autres fonctions des sous-modules. C'est le seul fichier du dossier modules/ qu'appelle le main. On peut dire qu'il sert d'interface entre le main.py et les autres modules.
	
	\end{itemize}

\underline{Niveau 2}
	\begin{itemize}
	\item \textbf{modules/ponctuation\_texte.py} :
	
	\item \textbf{modules/tagging.py} :
	
	\item \textbf{modules/stats\_0\_distributions.py} :
	
	\item \textbf{modules/stats\_1\_base.py} :
	
	\item \textbf{modules/writing\_in\_files.py} :

	\item \textbf{modules/others.py} :
	\end{itemize}

%-------------------------------------------------------------
\newpage
\section{Processus}

\begin{itemize}
\item \textbf{- [x]} : 
\item \textbf{- [x]} : 
\item \textbf{- [x]} : 
\item \textbf{- [x]} : 
\item \textbf{- [x]} : 
\item \textbf{- [x]} : 
\item \textbf{- [x]} : 
\item \textbf{- [x]} : 
\item \textbf{- [x]} : 
\item \textbf{- [x]} : 
\item \textbf{- [x]} : 
\item \textbf{- [x]} : 
\item \textbf{- [x]} : 
\item \textbf{- [x]} : 
\item \textbf{- [x]} : 
\item \textbf{- [x]} : 
\item \textbf{- [x]} : 
\item \textbf{- [x]} : 
\end{itemize}









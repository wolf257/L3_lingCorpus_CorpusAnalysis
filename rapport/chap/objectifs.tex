\section{Scénario} 

Une maison d'édition reçoit des centaines de manuscrits par semaine. Sur suggestion, elle fait appel à un programmeur en herbe pour créer un petit programme qui permettrait de faire un premier tri. \\

Spécialisée dans la littérature ancienne, elle dispose d'un corpus de référence, composé de textes de V. Hugo, A. Dumas, Montaigne, et Homère. Un jour, elle reçoit un ensemble de texte de notre professeur, Mr Paroubek. \\

\section{Matériaux}

Nous disposons donc de l'ensemble des cours de L3 de Mr Paroubek sur l'année 2016-17\footnote{disponible sur \emph{www.perso.limsi.fr/pap/inalco}}, ainsi que d'un corpus de texte littéraire de grands auteurs\footnote{Les textes sont disponibles sur \emph{www.gutemberg.org}}. \\

\section{Cahier de charge}

Il s'agit pour nous de créer un programme en langage python\footnote{Python3} capable, après normalisation des fichiers, d'effectuer des analyses statistiques de base sur un corpus (deux en fait), d'en faire l'étiquetage morpho-syntaxique, et d'exporter le résultat sous forme d'un fichier xml. \\

Mais aussi, nous voulons comparer les deux corpus pour voir si le corpus reçu passe le test statistique. \\

À priori, nous nous attendons à une différence nette entre un corpus littéraire et un corpus de diapositives destinées à des cours d'informatique. \\






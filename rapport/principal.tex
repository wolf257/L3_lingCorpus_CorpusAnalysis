        %////////////////////////////////////////////////////////////////
       %
     %		BEFORE DOCUMENT
   %
%////////////////////////////////////////////////////////////////

\documentclass[12pt, a4paper]{report}%, twoside, openright]{report}

         %////////////////////////////////////////////////////////////////
     %		PACKAGES
%////////////////////////////////////////////////////////////////

%////////////////////////////////////////////////////////////////
%		INPUT
%////////////////////////////////////////////////////////////////
%\usepackage[latin1]{inputenc} 
%\usepackage[T1]{fontenc} 
%\usepackage[francais]{babel} 
\usepackage[utf8]{luainputenc}
\usepackage{fontspec}
\usepackage{unicode-math}

%------------ Package "arabluatex"  ------------
\usepackage[novoc]{arabluatex}
\newfontfamily\arabicfont[Script=Arabic]{Scheherazade}
% Polices [Noto Naskh Arabic] | [Scheherazade]
% Options1 : voc, fullvoc, novoc, trans
% Morceau : \arb[options1] {...}
% Paragraphe : \begin{arab}[options1] ... \end{arab}
% Num au format abjad : \abjad{...}
% Parenthese (en plus des standards} : \abraces{...}

%----------------------------------------------------------------------
% Insertion arabe dans tableau (pas que!) (La taille est augmenté)
\newcommand{\arbt}[1]{\Large{\arb{#1}}} %{\arbt{...}}
\newcommand{\arbT}[1]{\LARGE\arb{#1}} %\arbT{...}
%----------------------------------------------------------------------

%////////////////////////////////////////////////////////////////
%		Mise en page (beg)
%////////////////////////////////////////////////////////////////

%------------ Package   ------------
\usepackage{layout}
\usepackage[top=2cm, bottom=2cm, left=2cm, right=2cm]{geometry} %marge

\usepackage{setspace} %package pour les interlignes
				 
\usepackage{multicol}
\setlength{\columnsep}{40pt}

\usepackage{xcolor} % package pour les couleurs 

	%////////////////////////////////////////////////////////////////
	%		Header and footer

\usepackage{fancyhdr} % header etc
\usepackage{fancybox} %package pour les encadrements supp
\pagestyle{fancy} % Use en-tetes et des pieds de page personnaliser grâce fancyhdr	

% ------------- PREMIERE PAGE DES CHAPITRES
	\fancypagestyle{plain}{
		\fancyhf{} % on efface tout 
	%	\fancyfoot[]{} 
	%	\fancyhead[]{} 
	%	% on efface tous les traits 
		\renewcommand{\headrulewidth}{0pt}
		\renewcommand{\footrulewidth}{0pt}
	}

% -------------PAGES GENERALES

% \fancyhead[args]{valeur} % définition en-tete
% \fancyfoot[args]{valeur} % définition pied de page
% Arg opt pour les deux : E, O for page's parity (even or odd)
% R, L, C : right, left or center

%\renewcommand{\headrulewidth}{...pt} % epaisseur trait horizontal en dessous de l’en-tête
%\renewcommand{\footrulewidth}{1pt} % epaisseur trait horizontal au dessus pied de page

%--\renewcommand{\chaptermark}[1]{\markboth{#1}{}}

\fancyhead[L]{}%Section  : \thesection ~ ...}%{Part : \thepart ~ Chap : \thechapter} % définition en-tête
%\fancyhead[R]{\leftmark} % définition en-tête
%\fancyfoot[L]{ } % définition pied de page
%\fancyfoot[R]{ {\small} } % définition pied de page

\usepackage[Glenn]{fncychap}
%Options: Sonny, Lenny, Glenn, Conny, Rejne, Bjarne, Bjornstrup


%////////////////////////////////////////////////////////////////
%		Mes commandes (beg)
%////////////////////////////////////////////////////////////////

\usepackage{float}
\usepackage{array, makecell} %pour les tableaux et makecell pour les cellules
\usepackage{multirow} %pr tableau

% Dans "tabular", une colonne de taille x, avec le texte centré, dans la definition, on utilisera P{taille} à la place de p{taille} ou c.
\newcolumntype{P}[1]{>{\centering\arraybackslash}m{#1}} %P{taille}

% Line-break in "tabular"
% \makecell{ ... \\ ... \\ ... }

 %////////////////////////////////////////////////////////////////
%		Mathematics and graphics 
%////////////////////////////////////////////////////////////////
	
\usepackage{amsmath, amsthm} %package pour les maths 
%inclut amsfonts, amssymb, mathtools, amstext

%----------------------------------------------------------------------
% def theoreme, definition, lemme, démonstration, preuve etc
\newtheorem{theorem}{Théorème}[section] 
\newtheorem{definition}{Définition}[section]
% il suffit d'appeler les environnements
% ex : \newtheorem{theorem|...}{Théorème|...}[section] % voir mes Macros
%----------------------------------------------------------------------

\usepackage{tikz} % package pour les graphiques
% use in environnement {tikzpicture}
\usetikzlibrary{shapes.geometric, arrows, trees} %to create flowcharts

%////////////////////////////////////////////////////////////////
%		...............
%////////////////////////////////////////////////////////////////

\usepackage{lettrine}
\usepackage{marvosym} %package de symbole

\usepackage{wrapfig} % pour les capsules in text

%////////////////////////////////////////////////////////////////
%		Autres (index, abstract)
%////////////////////////////////////////////////////////////////

%------------ Package imakeidx (multindex)  ------------
\usepackage{imakeidx} 
\makeindex %index standard
% indexer un mot : mot\index{nom}
% Print l'index : \printindex

% Creation autre index : \makeindex[name_in_text =... , title_in_pdf = ...]
% Indexer un mot : mot\index[n_i_t]{nom}
% Print l'index : \printindex[n_i_t]

%------------ Abstract (if required)  ------------
% Nom de l'abstract : 
%\renewcommand{\abstractname}{Executive Summary}

%ajouter juste après le onehalfspace
	%\clearpage{\pagestyle{empty}\cleardoublepage}
	%\begin{abstract}
		%Your abstract goes here...
	%\end{abstract}

%-------------- ABREVIATION (don't work yet !!!!)
%\usepackage[french]{nomencl}
%\makenomenclature
%pour declarer une abreviation :  \nomenclature{abrevation}{signification}
%pour creer la liste d'abreviation : \printnomenclature
 
 %To reset counter chapter with part
   \makeatletter
   \@addtoreset{chapter}{part}
   \makeatother  

 \usepackage{hyperref}
%\hypersetup{
  %  colorlinks=true, %set true if you want colored links
   % linktoc=all,     %set to all if you want both sections and subsections linked
    %linkcolor=blue,  %choose some color if you want links to stand out
%}
        %////////////////////////////////////////////////////////////////
     %		TITLE'S INFORMATIONS
%////////////////////////////////////////////////////////////////

\title{\textbf{Rapport de projet L3S5 : Linguistique de Corpus} }
\author{Ousseynou GUEYE}
\date{\today}

        %////////////////////////////////////////////////////////////////
       %
     %		DOCUMENT'S BEGINNING
   %
%////////////////////////////////////////////////////////////////
\begin{document}
\maketitle
	%\begin{twocolumn}
		\begin{onehalfspace}

\tableofcontents


    	%////////////////////////////////////////////////////////////////
	%		Start editing
	%////////////////////////////////////////////////////////////////

%-------------------------------------------------------------------
	\chapter{Définition des objectifs}
\section{Scénario} 

Une maison d'édition reçoit des centaines de manuscrits par semaine. Sur suggestion, elle fait appel à un programmeur en herbe pour créer un petit programme qui permettrait de faire un premier tri. \\

Spécialisée dans la littérature ancienne, elle dispose d'un corpus de référence, composé de textes de V. Hugo, A. Dumas, Montaigne, et Homère. Un jour, elle reçoit un ensemble de texte de notre professeur, Mr Paroubek. \\

\section{Matériaux}

Nous disposons donc de l'ensemble des cours de L3 de Mr Paroubek sur l'année 2016-17\footnote{disponible sur \emph{www.perso.limsi.fr/pap/inalco}}, ainsi que d'un corpus de texte littéraire de grands auteurs\footnote{Les textes sont disponibles sur \emph{www.gutemberg.org}}. \\

\section{Cahier de charge}

Il s'agit pour nous de créer un programme en langage python\footnote{Python3} capable, après normalisation des fichiers, d'effectuer des analyses statistiques de base sur un corpus (deux en fait), d'en faire l'étiquetage morpho-syntaxique, et d'exporter le résultat sous forme d'un fichier xml. \\

Mais aussi, nous voulons comparer les deux corpus pour voir si le corpus reçu passe le test statistique. \\

À priori, nous nous attendons à une différence nette entre un corpus littéraire et un corpus de diapositives destinées à des cours d'informatique. \\







%-------------------------------------------------------------------
	\chapter{Organisation des dossiers et du code}
\section{Approche}

%-------------------------------------------------------------
\newpage
\section{Organisation des dossiers}

\fbox{
\begin{minipage}[t]{.4\textwidth}
\footnotesize{ 
\begin{center}
AVANT EXÉCUTION DU PROGRAMME
\end{center}

\begin{itemize}
\item \textbf{PROJECT\_ROOT/}
	\begin{itemize}
	\item {code\_source/}
		\begin{itemize}
		\item {modules/}
		\end{itemize}
	\item {corpus\_litterature/}
	\item {corpus\_professeur/}
	\item {from\_outside\_treetagger/}
	\item {morphalo/}
	\item {rapport/}
	\end{itemize}
\end{itemize}

}
\end{minipage} 
} \fbox{
\begin{minipage}[t]{.4\textwidth}
\footnotesize{ 
\begin{center}
APRÈS EXÉCUTION DU PROGRAMME
\end{center}

\begin{itemize}
\item \textbf{PROJECT\_ROOT/}
	\begin{itemize}
	\item {code\_source/}
		\begin{itemize}
			\item {modules/}
		\end{itemize}
	\item {corpus\_litterature/}
		\begin{itemize}
		\item \textbf{statistiques/}
		\item \textbf{xml/}
		\item \textbf{pdf/}
		\end{itemize}
	\item {corpus\_professeur/}
		\begin{itemize}
		\item \textbf{statistiques/}
		\item \textbf{xml/}
		\end{itemize}
	\item {from\_outside\_treetagger/}
	\item {morphalo/}
	\item {rapport/}
	\item \textbf{resultats\_xml/}
	\end{itemize}
\end{itemize}
}
\end{minipage}
} 

~\\

Les noms nous paraissent assez explicites pour ne pas être tous détaillés. Néanmoins, certains dossiers nécessitent  une petite explication.

	\begin{itemize}
	\item \textbf{from\_outside\_treetagger/} : au début de la programmation, l'accès aux éléments de morphalou était difficile. Pour ne pas perdre de temps, nous avions décidé d'utiliser aussi tree\_tagger.
	
	Celui-ci ayant besoin de fichiers particuliers pour fonctionner, nous les avons regroupés sous ce dossier.
	
	\item \textbf{corpus$_x$/statistiques/} : Contient les statistiques de base, ainsi que les distributions de mots de chaque fichier pris séparément, puis du corpus en entier.
	
	\item \textbf{corpus$_x$/xml/} : Contient l'arbre xml concernant le corpus.
	
	\item \textbf{resultats\_xml} : S'appuyant sur les deux dossiers précédemment cités, ce dossier contient toutes les données que nous avons pu recueillir.
	\end{itemize}

%------------------------------------------------------------
\newpage
\section{Présentation des modules}

La partie programmée est regroupée dans le dossier \emph{code\_source}. \\

 \begin{wrapfigure}{l}{.45\textwidth} % { r, l, i } pour la position,
\fbox{
\begin{minipage}{.40\textwidth}
\footnotesize{
\textbf{code\_source/}
	\begin{itemize}
	\item {main.py}
	\item {settings.py}
	\item {modules/}
		\begin{itemize}
		\item {big\_process.py} \\
		\item {ponctuation\_texte.py}
		\item {tagging.py}
		\item {stats\_0\_distributions.py}
		\item {stats\_1\_base.py}
		
		\item {writing\_in\_files.py}
		\item {others.py}
		\end{itemize}
	\end{itemize}
}
\end{minipage}
} 
\end{wrapfigure}

Cette fois-ci, chaque fichier mérite que l'on vous explicite ce qu'il contient. \\

De même, nous aimerions préciser la logique de notre mouvement (flow). C'est en la suivant que j'ai pu, de manière systématique naviguer à travers le code\footnote{qui devenait de plus en plus gros.}. Je pars toujours du niveau 0 vers le niveau 2. \\  

\underline{Niveau 0}
	\begin{itemize}
	\item \textbf{main.py} : Contient l'interface. C'est le fichier qui englobe le tout.
	\item \textbf{settings.py} : Pour faciliter la gestion des liens, nous avons regroupé les liens vers nos dossiers dans ce fichier. Il ne contient aucune fonction.
	\end{itemize}

\underline{Niveau 1}
	\begin{itemize}
	\item \textbf{modules/big\_process.py} : Ne contient que des fonctions composées d'autres fonctions des sous-modules. C'est le seul fichier du dossier modules/ qu'appelle le main. On peut dire qu'il sert d'interface entre le main.py et les autres modules.
	
	\end{itemize}

\underline{Niveau 2}
	\begin{itemize}
	\item \textbf{modules/ponctuation\_texte.py} : 
	
	\item \textbf{modules/tagging.py} :
	
	\item \textbf{modules/stats\_0\_distributions.py} :
	
	\item \textbf{modules/stats\_1\_base.py} :
	
	\item \textbf{modules/writing\_in\_files.py} :

	\item \textbf{modules/others.py} :
	\end{itemize}

%-------------------------------------------------------------
\newpage
\section{Processus}

\begin{itemize}
\item \textbf{- [x]} : 
\item \textbf{- [x]} : 
\item \textbf{- [x]} : 
\item \textbf{- [x]} : 
\item \textbf{- [x]} : 
\item \textbf{- [x]} : 
\item \textbf{- [x]} : 
\item \textbf{- [x]} : 
\item \textbf{- [x]} : 
\item \textbf{- [x]} : 
\item \textbf{- [x]} : 
\item \textbf{- [x]} : 
\item \textbf{- [x]} : 
\item \textbf{- [x]} : 
\item \textbf{- [x]} : 
\item \textbf{- [x]} : 
\item \textbf{- [x]} : 
\item \textbf{- [x]} : 
\end{itemize}










%ss forme de liste
%-------------------------------------------------------------------
	\chapter{Difficultés rencontrées}
\section{...}
\section{...}
\section{...}


%lire la docu odododod

%-------------------------------------------------------------------
	\chapter{Résultats}
\section{Fichier xml final}

Le résultat du programme est inscrit dans le fichier xml qui se présente ainsi : \\ 

\fbox{
\begin{minipage}{.85\textwidth}
\footnotesize{

\texttt{<?xml version="1.0" encoding="UTF-8" standalone="no" ?>}

\texttt{<!DOCTYPE banque\_donnees SYSTEM "synthese\_xml\_total.dtd">} \\

\texttt{<banque\_donnees> } 

\texttt{\hspace{.5cm}<corpus type\_corpus='corpus\_*'> } 

\texttt{\hspace{1cm}<text text\_id='c.t.'> } 

\texttt{\hspace{1.5cm}<sentence sentence\_id='c.t.s.'> } 

\texttt{\hspace{2cm}<word word\_id='c.t.s.w.' word\_form='xxx'> } 

%----------------------------------------------------------------------------
\texttt{\hspace{2.5cm}<morphology> } 

\texttt{\hspace{2.8cm}<lemme> } 
\texttt{\hspace{1cm}<![CDATA[ x ]]>}
\texttt{\hspace{1cm}</lemme> } 

\texttt{\hspace{2.8cm}<POS\_treetagger> } 
\texttt{\hspace{0cm}<![CDATA[ x ]]>}
\texttt{\hspace{0cm}</POS\_treetagger> } 

\texttt{\hspace{2.8cm}<POS\_morphalou> } 
\texttt{\hspace{0cm}<![CDATA[ x ]]>}
\texttt{\hspace{0cm}</POS\_morphalou> } 

\texttt{\hspace{2.8cm}<genre> } 
\texttt{\hspace{0cm}<![CDATA[ x ]]>}
\texttt{\hspace{0cm}</genre> } 

\texttt{\hspace{2.8cm}<nombre> } 
\texttt{\hspace{0cm}<![CDATA[ x ]]>}
\texttt{\hspace{0cm}</nombre> } 

\texttt{\hspace{2.5cm}</morphology> } 

%----------------------------------------------------------------------------
\texttt{\hspace{2.5cm}<statistics> } 

\texttt{\hspace{2.8cm}<nb\_apparition\_text> } 
\texttt{\hspace{0cm}<![CDATA[ x ]]>}
\texttt{\hspace{0cm}</nb\_apparition\_text> } 

\texttt{\hspace{2.8cm}<nb\_apparition\_corpus> } 
\texttt{\hspace{0cm}<![CDATA[ x ]]>}
\texttt{\hspace{0cm}</nb\_apparition\_corpus> } 

\texttt{\hspace{2.8cm}<frequence\_in\_text> } 
\texttt{\hspace{0cm}<![CDATA[ x ]]>}
\texttt{\hspace{0cm}</frequence\_in\_text> } 

\texttt{\hspace{2.8cm}<frequence\_in\_corpus> } 
\texttt{\hspace{0cm}<![CDATA[ x ]]>}
\texttt{\hspace{0cm}</frequence\_in\_corpus> } 

\texttt{\hspace{2.5cm}</statistics> } 
%----------------------------------------------------------------------------

\texttt{\hspace{2cm}</word> } 

\texttt{\hspace{1.5cm}</sentence> } 

\texttt{\hspace{1cm}</text> } 

\texttt{\hspace{.5cm}</corpus> } 

\texttt{</banque\_donnees> } 

}
\end{minipage}
}

~~\\

Chaque élément xml (du corpus au mot) peut être multiplié autant de fois que nécessaire. \\

\section{Temps d'exécution}

Nous avons testé le programme avec un nombre de phrases à traiter différent\footnote{Si le nombre de phrases à traiter excède le nombre de phrases contenues dans le fichier, la variable s'ajuste automatiquement au nombre de phrases du fichier}. \\

\begin{tabular}{|c|c|}
\hline
\textbf{Nombre de phrases par fichier} & \textbf{Temps d'exécution approximatif} \\ 
\hline \hline
1 phrase & 3 minutes \\
\hline
10 phrases & 20 minutes \\
\hline
15 phrases & 30 minutes \\
\hline
50 phrases & 1 heure 30 minutes \\
\hline
\end{tabular} 
~\\

%----------------------------------------------------------------------------
\section{Réponse à notre hypothèse initiale}

Comme dit au point 1.3, il nous fallait voir si le corpus de Mr Paroubek passait notre premier texte. \\

Contrairement à ce que l'on pensait, au niveau du nombre de mots par phrase\footnote{qui était notre principal outil de séparation}, la différence n'est pas significative : \\

\begin{tabular}{|c|P{3cm}|P{4cm}|P{3cm}|}
\hline
\textbf{Corpus} & \textbf{Nombre de mots total} & \textbf{Nombre de mots par phrase} & \textbf{Différence} \\ 
\hline \hline
Corpus littérature & 2.570.695 & 22.80 & + 8 \% \\
\hline
Corpus Mr Paroubek & 23.163 & 21.07 & - 8 \% \\
\hline
\end{tabular} 
~\\

Ainsi, basé sur ce critère, le corpus de Mr Paroubek passe le premier test. \\


	\chapter{Limites}
%\input{chap/}
%test sur qqes centaines de phrases, machine pas assez puissante

%-------------------------------------------------------------------
	\chapter{Ouverture}
%\input{chap/}

%ss forme de tableau

%-------------------------------------------------------------------
	\chapter{}




  	 %////////////////////////////////////////////////////////////////
	 %		End editing
	%////////////////////////////////////////////////////////////////

\newpage

\tableofcontents

          %////////////////////////////////////////////////////////////////
       %
     %		DOCUMENT'S END
   %
%////////////////////////////////////////////////////////////////

		\end{onehalfspace}
	%\end{twocolumn}
\end{document}

%------------------------------ Fin document ---------------------------------
